%% This document created by Scientific Notebook (R) Version 3.0


\documentclass[12pt,thmsa]{article}
%%%%%%%%%%%%%%%%%%%%%%%%%%%%%%%%%%%%%%%%%%%%%%%%%%%%%%%%%%%%%%%%%%%%%%%%%%%%%%%%%%%%%%%%%%%%%%%%%%%%%%%%%%%%%%%%%%%%%%%%%%%%
\usepackage{sw20jart}

%TCIDATA{TCIstyle=article/art4.lat,jart,sw20jart}

%TCIDATA{<META NAME="GraphicsSave" CONTENT="32">}
%TCIDATA{Created=Mon Aug 19 14:52:24 1996}
%TCIDATA{LastRevised=Mon Apr 04 00:13:12 2005}
%TCIDATA{CSTFile=Lab Report.cst}
%TCIDATA{PageSetup=72,72,72,72,0}
%TCIDATA{AllPages=
%F=36,\PARA{038<p type="texpara" tag="Body Text" >\hfill \thepage}
%}


\input{tcilatex}
\begin{document}


\subsection{2--7-stc}

\subsection{Sebasti\'{a}n Torrente Carrillo:}

\vspace{1pt}\textbf{Calcular la resistencia equivalente en un sistema
tridimensional de resistencias (con una de ellas variable):}

Dado el el cubo de resistencias:

\begin{center}
\FRAME{dtbpF}{216.25pt}{191.9375pt}{0pt}{}{}{Figure }{\special{language
"Scientific Word";type "GRAPHIC";maintain-aspect-ratio TRUE;display
"USEDEF";valid_file "T";width 216.25pt;height 191.9375pt;depth
0pt;original-width 213.0625pt;original-height 188.9375pt;cropleft
"0";croptop "1";cropright "1";cropbottom "0";tempfilename
'IEE5Q000.wmf';tempfile-properties "XPR";}}
\end{center}

Se nos pide calcular la resistencia equivalente y los valores para dicha
resistencia, en el caso en el que todas las resistencias tengan un mismo
valor (1 ohm) salvo, la resistencia ubicada entre los puntos 1y 2 (que
ser\'{a} variable).

En el informe correspondiente a la pr\'{a}ctica 2-6-stc se detalla el metodo
para obtener una resistencia equivalente mediante el uso de las Leyes de
Kirchoff y de el c\'{a}lculo de ecuaciones lineales. Por lo que no
repetiremos lo dicho all\'{i}.

Sin embargo este caso tiene importantes diferencias con respecto al
anterior. La m\'{a}s obvia es la disposici\'{o}n tridimensional de la red,
la otra diferencia es la resistencia variable que tenemos en el conjunto.

(Recordemos que tenemos como dato el que el valor de la resistencia
equivalente en el caso de que todas tengan un mismo valor R es de $\frac{5R}{%
6}$ que en nuestro caso ser\'{i}a simplemente $\frac{5}{6}$)

Esto nos da una de idea de como abordar el problema. Resolveremos este
problema igual que el anterior, solo que ahora una de las resistencias del
sistema ir\'{a} tomando distintos valores dentro de un intervalo.
Obtendremos una sucesi\'{o}n de valores de R que podremos representar
graficamente.

\vspace{1pt}

El sistema de matrices correspondiente al problema es el siguiente:

$\left( 
\begin{array}{lllllllllllll}
1 & 0 & 0 & 1 & 0 & 0 & 1 & 0 & 0 & 0 & 0 & 0 & -1 \\ 
1 & -1 & 0 & 0 & 0 & 0 & 0 & -1 & 0 & 0 & 0 & 0 & 0 \\ 
0 & 0 & -1 & 1 & 0 & 0 & 0 & -1 & 0 & 0 & 0 & 0 & 0 \\ 
0 & 1 & 1 & 0 & 0 & -1 & 0 & 0 & 0 & 0 & 0 & 0 & 0 \\ 
0 & 0 & 0 & 0 & 0 & 0 & -1 & 0 & -1 & 1 & 0 & 0 & 0 \\ 
0 & 0 & 0 & 0 & -1 & 0 & 0 & 0 & 0 & -1 & 1 & 0 & 0 \\ 
0 & 0 & 0 & 0 & 0 & 0 & 0 & -1 & 1 & 0 & 0 & 1 & 0 \\ 
R_{1} & R_{2} & -R_{3} & -R_{4} & 0 & 0 & 0 & 0 & 0 & 0 & 0 & 0 & 0 \\ 
0 & 0 & 0 & -R_{4} & 0 & 0 & R_{7} & -R_{8} & -R_{9} & 0 & 0 & 0 & 0 \\ 
R_{1} & 0 & 0 & 0 & R_{5} & 0 & R_{7} & 0 & 0 & -R_{10} & 0 & 0 & 0 \\ 
0 & -R_{2} & 0 & 0 & R_{5} & -R_{6} & 0 & 0 & 0 & 0 & R_{11} & 0 & 0 \\ 
0 & 0 & 0 & 0 & 0 & 0 & 0 & 0 & R_{9} & R_{10} & R_{11} & -R_{12} & 0 \\ 
R_{1} & 0 & 0 & 0 & R_{5} & 0 & 0 & 0 & 0 & 0 & R_{11} & 0 & 0
\end{array}
\right) \cdot \left( 
\begin{array}{l}
I_{1} \\ 
I_{2} \\ 
I_{3} \\ 
I_{4} \\ 
I_{5} \\ 
I_{6} \\ 
I_{7} \\ 
I_{8} \\ 
I_{9} \\ 
I_{10} \\ 
I_{11} \\ 
I_{12} \\ 
I_{13}
\end{array}
\right) =\left( 
\begin{array}{l}
0 \\ 
0 \\ 
0 \\ 
0 \\ 
0 \\ 
0 \\ 
0 \\ 
0 \\ 
0 \\ 
0 \\ 
0 \\ 
0 \\ 
V
\end{array}
\right) $

Donde todas las resitencias tienen un valor de 1 ohm, excepto R1, que tiene
valor variable.

\end{document}
