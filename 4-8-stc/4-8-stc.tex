%% This document created by Scientific Notebook (R) Version 3.0


\documentclass[12pt,thmsa]{article}
%%%%%%%%%%%%%%%%%%%%%%%%%%%%%%%%%%%%%%%%%%%%%%%%%%%%%%%%%%%%%%%%%%%%%%%%%%%%%%%%%%%%%%%%%%%%%%%%%%%%%%%%%%%%%%%%%%%%%%%%%%%%
\usepackage{sw20jart}

%TCIDATA{TCIstyle=article/art4.lat,jart,sw20jart}

%TCIDATA{<META NAME="GraphicsSave" CONTENT="32">}
%TCIDATA{Created=Mon Aug 19 14:52:24 1996}
%TCIDATA{LastRevised=Thu Apr 28 23:16:54 2005}
%TCIDATA{CSTFile=Lab Report.cst}
%TCIDATA{PageSetup=72,72,72,72,0}
%TCIDATA{AllPages=
%F=36,\PARA{038<p type="texpara" tag="Body Text" >\hfill \thepage}
%}


\input{tcilatex}
\begin{document}


\subsubsection{\protect\vspace{1pt}Sebasti\'{a}n Torrente Carrillo}

\subsubsection{b) 4-8 Estudiando la forma de realizar predicciones metereol%
\'{o}gicas, Edward N. Lorenz public\'{o} en 1963 el siguiente sistema de
ecuaciones diferenciales de primer orden acopladas:}

\subsubsection{$\frac{dx}{dt}=\sigma (y-x)$}

\subsubsection{$\frac{dy}{dt}=rx-y-zx$}

\subsubsection{$\frac{dz}{dt}=xy-bz$}

\subsubsection{Con $\sigma ,b,r$ constantes positivas.}

\subsubsection{Resuelve numericamente este problema para el intervalo desde
t=0 hasta t=30 unidades arbitrarias de tiempo. Emplea los valores iniciales
(1,1,20,0002) y (1,1,20,0009):}

\vspace{1pt}

a) Representa con los valores iniciales que se te han dado:

Primero representaremos los datos correspondientes a las primeras
condiciones iniciales:

$x(t):$

\FRAME{dtbpF}{512.75pt}{257.75pt}{0pt}{}{}{Figure }{\special{language
"Scientific Word";type "GRAPHIC";maintain-aspect-ratio TRUE;display
"USEDEF";valid_file "T";width 512.75pt;height 257.75pt;depth
0pt;original-width 609.75pt;original-height 305.625pt;cropleft "0";croptop
"1";cropright "1";cropbottom "0";tempfilename
'IFOD8A06.wmf';tempfile-properties "XPR";}}

$y(z)$ (diagrama de fases):

\FRAME{dtbpF}{514.25pt}{258.5625pt}{0pt}{}{}{Figure }{\special{language
"Scientific Word";type "GRAPHIC";maintain-aspect-ratio TRUE;display
"USEDEF";valid_file "T";width 514.25pt;height 258.5625pt;depth
0pt;original-width 609.75pt;original-height 305.625pt;cropleft "0";croptop
"1";cropright "1";cropbottom "0";tempfilename
'IFODB307.wmf';tempfile-properties "XPR";}}

Para las segundas condiciones iniciales:

\FRAME{dtbpF}{512.75pt}{257.8125pt}{0pt}{}{}{Figure }{\special{language
"Scientific Word";type "GRAPHIC";maintain-aspect-ratio TRUE;display
"USEDEF";valid_file "T";width 512.75pt;height 257.8125pt;depth
0pt;original-width 609.75pt;original-height 305.625pt;cropleft "0";croptop
"1";cropright "1";cropbottom "0";tempfilename
'IFODDM08.wmf';tempfile-properties "XPR";}}

\FRAME{dtbpF}{514.25pt}{258.5625pt}{0pt}{}{}{Figure }{\special{language
"Scientific Word";type "GRAPHIC";maintain-aspect-ratio TRUE;display
"USEDEF";valid_file "T";width 514.25pt;height 258.5625pt;depth
0pt;original-width 609.75pt;original-height 305.625pt;cropleft "0";croptop
"1";cropright "1";cropbottom "0";tempfilename
'IFODEK09.wmf';tempfile-properties "XPR";}}

\vspace{1pt}

\textquestiondown Que puedes concluir de tus resultados?

La variaci\'{o}n de los valores es muy peque\~{n}a, sin embargo la variaci%
\'{o}n de los resultados es muy grande, si adem\'{a}s hacemos la comprobaci%
\'{o}n con unos valores que divergan a\'{u}n m\'{a}s de los que nos han dado
(tomemos por ejemplo x=2, y=10, z=5), comprobaremos que los resultados
divergen a\'{u}n m\'{a}s:

\FRAME{dtbpF}{399.8125pt}{201pt}{0pt}{}{}{Figure }{\special{language
"Scientific Word";type "GRAPHIC";maintain-aspect-ratio TRUE;display
"USEDEF";valid_file "T";width 399.8125pt;height 201pt;depth
0pt;original-width 609.75pt;original-height 305.625pt;cropleft "0";croptop
"1";cropright "1";cropbottom "0";tempfilename
'IFODLN0A.wmf';tempfile-properties "XPR";}}

\FRAME{dtbpF}{401.3125pt}{201.75pt}{0pt}{}{}{Figure }{\special{language
"Scientific Word";type "GRAPHIC";maintain-aspect-ratio TRUE;display
"USEDEF";valid_file "T";width 401.3125pt;height 201.75pt;depth
0pt;original-width 609.75pt;original-height 305.625pt;cropleft "0";croptop
"1";cropright "1";cropbottom "0";tempfilename
'IFODM80B.wmf';tempfile-properties "XPR";}}

Como se se\~{n}ala en el enunciado esta ecuaci\'{o}n se enunci\'{o} mientras
se estudiaban los metodos para realizar predicciones meteorolog\'{i}cas. Lo
que esta ecuaci\'{o}n revela es la extrema dificultad que eso implica, ya
que como se puede ver, la m\'{a}s m\'{i}nima variaci\'{o}n puede alterar
mucho los resultados. Eso se debe al hecho de que la cantidad de factores
que pueden influir en el clima es gigantesca, de ah\'{i} lo impredecible de
sus resultados.

\end{document}
