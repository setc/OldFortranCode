%% This document created by Scientific Notebook (R) Version 3.0


\documentclass[12pt,thmsa]{article}
%%%%%%%%%%%%%%%%%%%%%%%%%%%%%%%%%%%%%%%%%%%%%%%%%%%%%%%%%%%%%%%%%%%%%%%%%%%%%%%%%%%%%%%%%%%%%%%%%%%%%%%%%%%%%%%%%%%%%%%%%%%%
\usepackage{sw20jart}

%TCIDATA{TCIstyle=article/art4.lat,jart,sw20jart}

%TCIDATA{<META NAME="GraphicsSave" CONTENT="32">}
%TCIDATA{Created=Mon Aug 19 14:52:24 1996}
%TCIDATA{LastRevised=Thu Jun 30 22:25:24 2005}
%TCIDATA{CSTFile=Lab Report.cst}
%TCIDATA{PageSetup=72,72,72,72,0}
%TCIDATA{AllPages=
%F=36,\PARA{038<p type="texpara" tag="Body Text" >\hfill \thepage}
%}


\input{tcilatex}
\begin{document}


\subsubsection{Practica III: Difraci\'{o}n}

\subsubsection{Sebasti\'{a}n Torrente Carrillo}

\vspace{1pt}

\textbf{OBJETIVOS:}

\vspace{1pt}

i ) Estimar el tama\~{n}o de una abertura circular a partir del tama\~{n}o
del disco de Airy que produce.

ii ) Estimar la focal de una lente de Fresnel buscando la posici\'{o}n de su
foco. Con \'{e}ste dato y conociendo la longitud de onda utilizada y el
tama\~{n}o de la lente, obtener el orden de la lente difractiva.

iii ) Obtener los datos de una red de difracci\'{o}n que act\'{u}a por
transmisi\'{o}n, midiendo la posici\'{o}n de los diferentes \'{o}rdenes
difractados. Realizar la misma operaci\'{o}n para otra red que act\'{u}a por
reflexi\'{o}n.

iv ) Utilizar la red de difracci\'{o}n que act\'{u}a por transmisi\'{o}n y
colocarla en el goni\'{o}metro en incidencia normal para separar las
l\'{i}neas espectrales. Una vez hecho esto medir el \'{a}ngulo relativo a la
incidencia normal para averiguar la longitud de onda de cada una de las
l\'{i}neas espectrales.

\vspace{1pt}

\textbf{BASE TE\'{O}RICA:}

\vspace{1pt}

Si producimos un disco de Airy a partir de una abertura circular podemos,
midiendo el tama\~{n}o del disco de Airy, estimar el tama\~{n}o de la
abertura circular, empleando la f\'{o}rmula:

$R_{Medio}=\frac{1.22\lambda D}{r}$

En este caso la focal ser\'{a} la distancia abertura-pantalla.\vspace{1pt} Y
con $\lambda $ longitud de onda de la luz, D la distancia abertura-pantalla,
r el radio de la abertura y $R_{Medio}$ la media del radio externo con el
radio interno del primer anillo oscuro.

Para averiguar el orden de la lente de Fresnel empleamos la f\'{o}rmula: $N=%
\frac{r^{2}}{\lambda f^{\prime }}$, con r el radio de la lente, f' la focal
y $\lambda $ la longitud de onda de la luz. Podemos medir r, f'; puesto que $%
\lambda $ es conocida se puede despejar N.

Para la red de difracci\'{o}n necesitaremos la ecuaci\'{o}n: $2d\sin (\theta
)=m\lambda $ que es la ecuaci\'{o}n de la red, con d la distancia entre
aberturas, la desviaci\'{o}n angular $\theta $ y m el orden de
difracci\'{o}n.

Una vez que se tiene los datos de la red de difracci\'{o}n se pondr\'{a} en
el goni\'{o}metro con una luz con una longitud de onda determinada, y
calcular las longitudes de onda dependiente de la desviaci\'{o}n angular.

\textbf{MONTAJE:}

\vspace{1pt}

El montaje utilizado para determinar el tama\~{n}o de la abertura:

$\cdot$ 1 banco \'{o}ptico

$\cdot $ 1 l\'{a}ser de He-Ne de $\lambda $=632 nm

$\cdot$ 1 abertura circular

$\cdot$ 1 pantalla

\vspace{1pt}\FRAME{dtbpF}{355.9375pt}{101.25pt}{0pt}{}{}{Figure }{\special%
{language "Scientific Word";type "GRAPHIC";maintain-aspect-ratio
TRUE;display "USEDEF";valid_file "T";width 355.9375pt;height 101.25pt;depth
0pt;original-width 441.875pt;original-height 124.1875pt;cropleft "0";croptop
"1";cropright "1";cropbottom "0";tempfilename
'IIWYQP0J.wmf';tempfile-properties "XPR";}}

\vspace{1pt}El montaje para estimar la focal de la lente.

$\cdot$ 1 banco \'{o}ptico

$\cdot$ 1 l\'{a}ser de He-Ne de $\lambda $=632 nm

$\cdot$ 1 lente zonal de Fresnel de radio r =1.975 cm 

$\cdot$ 1 lente colimadora

$\cdot$ 1 pantalla

\FRAME{dtbpF}{346.875pt}{114.6875pt}{0pt}{}{}{Figure }{\special{language
"Scientific Word";type "GRAPHIC";maintain-aspect-ratio TRUE;display
"USEDEF";valid_file "T";width 346.875pt;height 114.6875pt;depth
0pt;original-width 434.375pt;original-height 142.3125pt;cropleft "0";croptop
"1";cropright "1";cropbottom "0";tempfilename
'IIWZ2K0K.wmf';tempfile-properties "XPR";}}

Caracterizaci\'{o}n de una red de difracci\'{o}n:

$\cdot$ 1 banco \'{o}ptico

$\cdot$ 1 l\'{a}ser de He-Ne de $\lambda $=632 nm

$\cdot$ 1 red de difracci\'{o}n que act\'{u}a por transmisi\'{o}n

$\cdot$ 1 red de difracci\'{o}n que act\'{u}a por reflexi\'{o}n

$\cdot$ 1 pantalla

\FRAME{dtbpF}{374.1875pt}{103.6875pt}{0pt}{}{}{Figure }{\special{language
"Scientific Word";type "GRAPHIC";maintain-aspect-ratio TRUE;display
"USEDEF";valid_file "T";width 374.1875pt;height 103.6875pt;depth
0pt;original-width 475pt;original-height 130.25pt;cropleft "0";croptop
"1";cropright "1";cropbottom "0";tempfilename
'IIWZBT0L.wmf';tempfile-properties "XPR";}}

Separaci\'{o}n de lineas espectrales:

-Simplemente colocamos la red de difracci\'{o}n sobre el espectrogoni\'{o}%
metro, con el que iremos buscando las lineas espectrales.

\textbf{DESARROLLO:}

\vspace{1pt}Determinar el tama\~{n}o de una abertura a partir de la mancha
de Airy:

Alineamos el sistema y colocamos la pantalla perpendicular al l\'{a}ser.
Medimos la distancia entre anillos.

Estimar la focal de una lente de Fresnel:

Primero colimamos el laser (el proceso es m\'{a}s facil si se emplea un
espejo), para medir la focal se coloca la pantalla en el punto en que que
convergen los rayos.

Obtener los datos de una red de difracci\'{o}n:

Como en los dem\'{a}s dispositivos experimentales, primero se alinean los
elementos del sistema, situando la pantalla a una distancia fija. En este
apartado se tiene que medir el angulo de salida de cada uno de los rayos
difractados. Si sabemos la distancia de la pantalla a la red podremos
calcular el \'{a}ngulo que se desvia cada haz, y con eso podremos
caracterizar la red.

Observar las lineas espectrales producidas por una red de difracci\'{o}n en
un espectrogoni\'{o}metro:

Es recomendable en esta pr\'{a}ctica emplear la m\'{i}nima iluminaci\'{o}n
posible (si es posible eliminar toda iluminaci\'{o}n del laboratorio), ya
que algunas lineas son muy tenues. Situamos la red en el soporte de
espectrogoni\'{o}metro, localizamos las lineas y medimos el \'{a}ngulo de
cada linea, rese\~{n}ando el color de cada linea.

\vspace{1pt}

\textbf{RESULTADOS:}

-Estimar el radio de la abertura:

$
\begin{array}{llllll}
\text{Distancia a la pantalla: }D(cm) & 186.8 & 176.8 & 156.8 & 126.8 & 106.8
\\ 
\text{Radio externo del anillo interno }R(cm) & 1.8 & 1.7 & 1.4 & 1.3 & 1.2
\\ 
\text{Radio interno del anillo externo }R^{\prime }(cm) & 2.5 & 2.3 & 2.1 & 
1.8 & 1.6 \\ 
\text{Radio medio: }R_{med} & 2.2 & 2.0 & 1.8 & 1.6 & 1.3
\end{array}
$

Podemos a partir de la ecuaci\'{o}n: $R_{Medio}=\frac{1.22\lambda D}{r}$,
respejar r y calcularla mediante un ajuste lineal:

Tenemos el resultado $r=73\pm 7\mu m$

-Focal de la lente de Fresnel:

$
\begin{array}{llllll}
f^{\prime }(cm)\pm 0.5 & 8.2 & 9.2 & 10.0 & 11.7 & 11.7 \\ 
N\pm 60 & 7350 & 6730 & 6170 & 5250 & 5250
\end{array}
$

Como se puede medir directamente, se toman varias medidas de la focal,
tomaremos como focal el valor medio, y a partir de esa focal podemos estimar
el orden de la lente:

$f^{\prime }(cm)=10.0\pm 0.7\longrightarrow N=6200\pm 900$

-Caracterizaci\'{o}n de una red de difracci\'{o}n:

-Por transmisi\'{o}n:

$
\begin{array}{llll}
\text{Orden: M} & -1 & 0 & 1 \\ 
\text{Posici\'{o}n del m\'{a}ximo: x(cm)}\pm 0.2 & -8.5 & 1.0 & 10.3 \\ 
\text{\'{A}ngulo }\theta (\UNICODE[m]{0xb0}) & -19.1\pm 0.7 & 0.04\pm 0.02 & 
3.88\pm 0.01
\end{array}
$

La ecuaci\'{o}n que caracteriza a una red se ha detallado en el apartado
correspondiente y es:

$2d\sin (\theta )=m\lambda $

Podemos despejar d y realizar un ajuste a una linea recta, obteniendo el
resultado:

$d=169\pm 80nm$

- Por reflexi\'{o}n:

\vspace{1pt}$
\begin{array}{llllll}
\text{Orden: M} & -2 & -1 & 0 & 1 & 2 \\ 
\text{Posici\'{o}n del m\'{a}ximo: x(cm)}\pm 0.5 & -23.4 & -7.9 & 0.0 & 8.8
& 25.0 \\ 
\text{\'{A}ngulo: } & -50.0\pm 1.4 & -22.0\pm 1.0 & 0.0\pm 1.0 & 24.0\pm 1.2
& 51.0\pm 1.4
\end{array}
$

Operando como en el caso anterior:

$d=820\pm 10nm$

\vspace{1pt}

-Lineas espectrales:

$
\begin{array}{llllllllll}
Posici\acute{o}n\text{ }M & 9 & 8 & 7 & 6 & 5 & 4 & 3 & 2 & 1 \\ 
Color & verde & azul & violeta & rojo & verde(claro) & verdeazulado & 
azul(claro) & violeta(claro) & violeta(tenue) \\ 
\acute{A}ngulo\text{ }\theta (\UNICODE[m]{0xb0}) & 35.5 & 33.0 & 32.0 & 22.0
& 18.0 & 17.0 & 16.5 & 16 & 15.5 \\ 
Longitud\text{ }de\text{ }onda\text{ }\lambda & 20.0 & 22.0 & 24.0 & 19.9 & 
19.7 & 23.4 & 30.3 & 44.1 & 85.5
\end{array}
\begin{array}{l}
0 \\ 
azul(claro) \\ 
0
\end{array}
$

$
\begin{array}{llllllllll}
Posici\acute{o}n\text{ }M & -9 & -8 & -7 & -6 & -5 & -4 & -3 & -2 & -1 \\ 
Color & verde & violeta & azul & rojo & verde(tenue) & verde & azul(claro) & 
azul(marino) & violeta(claro) \\ 
\acute{A}ngulo\text{ }\theta (\UNICODE[m]{0xb0}) & -35.0 & -33.0 & -32.0 & 
-21 & -18 & -17.0 & -16.0 & -1.5 & -1.4 \\ 
Longitud\text{ }de\text{ }onda\text{ }\lambda & 20.4 & 20.8 & 24.3 & 19.1 & 
19.7 & 23.4 & 29.4 & 41.4 & 77.4
\end{array}
$

\textbf{CONCLUSIONES:}

-Se puede apreciar que la precisi\'{o}n de las medidas obtenidas dejan
bastante que desear, algo que se puede ver en los resutados que hemos
obtenido.

-En el m\'{e}todo de medici\'{o}n del disco de Airy si se han obtenido
buenos resultados, debido a la facilidad de su realizaci\'{o}n. Y\ un mejor
m\'{e}todo para medir la desviaci\'{o}n angular.

-Hay que se\~{n}alar el enorme error relativo a la hora de medir el radio en
la primera parte, a pesar de que la medida es precisa.

-No hay una buena correspondencia en la longitud de onda de las lineas
espectrales, lo que quiere decir que la red de difracci\'{o}n no est\'{a}
bien caracterizada.

-Las razones por las que la red no est\'{a} bien caracterizada pueden ser
las siguientes:

\qquad -Pantalla no perpendicular.

\qquad\ -Medida de la distancia de cada orden al centro (especialmente
dificil en el caso de la reflexi\'{o}n).

\ \ \ \ \ \ -Laser mal colimado.

\vspace{1pt}

\end{document}
