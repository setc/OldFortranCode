%% This document created by Scientific Notebook (R) Version 3.0


\documentclass[12pt,thmsa]{article}
%%%%%%%%%%%%%%%%%%%%%%%%%%%%%%%%%%%%%%%%%%%%%%%%%%%%%%%%%%%%%%%%%%%%%%%%%%%%%%%%%%%%%%%%%%%%%%%%%%%%%%%%%%%%%%%%%%%%%%%%%%%%
\usepackage{sw20jart}

%TCIDATA{TCIstyle=article/art4.lat,jart,sw20jart}

%TCIDATA{<META NAME="GraphicsSave" CONTENT="32">}
%TCIDATA{Created=Mon Aug 19 14:52:24 1996}
%TCIDATA{LastRevised=Thu Jun 30 22:07:30 2005}
%TCIDATA{CSTFile=Lab Report.cst}
%TCIDATA{PageSetup=72,72,72,72,0}
%TCIDATA{AllPages=
%F=36,\PARA{038<p type="texpara" tag="Body Text" >\hfill \thepage}
%}


\input{tcilatex}
\begin{document}


\subsubsection{Practica 2: Interferencias.}

\subsubsection{Sebasti\'{a}n Torrente Carrillo.}

\textbf{OBJETIVOS:}

\vspace{1pt}

i ) Utilizar la separaci\'{o}n entre m\'{a}ximos en la configuraci\'{o}n
interferencial que dos rendijas producen, para deducir la separaci\'{o}n
entre \'{e}stas.

ii ) Utilizar el dato obtenido en el dato anterior para deducir la longitud
de onda de un l\'{a}ser de longitud de onda desconocida.

iii ) Representar la configuraci\'{o}n interferencial que producen una, dos
y tres rendijas.

iv ) Utilizar el interfer\'{o}metro de Michelson para comprobar la
precisi\'{o}n del tornillo microm\'{e}trico que regula el espejo m\'{o}vil
del interfer\'{o}metro.

\textbf{BASE TE\'{O}RICA:}

\vspace{1pt}

\vspace{1pt}Si un haz de luz de longitud de onda $\lambda $ incide sobre dos
rendijas y colocamos una pantalla para observar el patr\'{o}n que forma el
haz de luz, tenemos que la luz interfiere de forma constructiva en unas
zonas de la pantalla y de manera destructiva en otras (esto se debe a la
naturaleza ondulatoria de la luz).

Se puede considerar (para simplificar) que la distancia entre las rendijas
es mucho menor que la distancia de estas a la pantalla (condici\'{o}n que es
muy facil que se cumpla) $d<<D$. Con esto podemos calcular con facilidad la
diferencia de camino entre la luz que sale de cada rendija. Empleando esta
diferencia de caminos podemos obtener la relaci\'{o}n de la distancia entre
m\'{a}ximos dependiento de la distancia entre las rendijas y entre las
rendijas y la pantalla:

\vspace{1pt}

$\Delta x=\frac{\lambda }{h}D$

\vspace{1pt}

\textbf{MONTAJE:}

\vspace{1pt}

Experimento de doble rendija de Young consta de:

$\cdot$ 1 l\'{a}ser de $\lambda $=650 nm

$\cdot$ 1 banco \'{o}ptico

$\cdot$ 1 l\'{a}mina de una, dos y tres rendijas.

$\cdot$ 1 pantalla

\vspace{1pt}\FRAME{dtbpF}{368.3125pt}{100pt}{0pt}{}{}{Figure }{\special%
{language "Scientific Word";type "GRAPHIC";maintain-aspect-ratio
TRUE;display "USEDEF";valid_file "T";width 368.3125pt;height 100pt;depth
0pt;original-width 496.125pt;original-height 133.25pt;cropleft "0";croptop
"1";cropright "1";cropbottom "0";tempfilename
'IIWY0F02.wmf';tempfile-properties "XPR";}}Interfer\'{o}metro de Michelson
consta de:

$\cdot$ 1 l\'{a}ser de $\lambda $=633 nm

$\cdot$ 1 lente de focal corta 

$\cdot$ 1 interfer\'{o}metro de Michelson

-1 espejo (que es parte del interfer\'{o}metro, pero no se encuentra en el
banco).

\FRAME{dtbpF}{367.25pt}{220.75pt}{0pt}{}{}{Figure }{\special{language
"Scientific Word";type "GRAPHIC";maintain-aspect-ratio TRUE;display
"USEDEF";valid_file "T";width 367.25pt;height 220.75pt;depth
0pt;original-width 426.125pt;original-height 255.375pt;cropleft "0";croptop
"1";cropright "1";cropbottom "0";tempfilename
'IIWY5C03.wmf';tempfile-properties "XPR";}}

En la foto se puede ver a la izquierda la lente de focal corta. A la derecha
se encontrar\'{i}a el espejo del que se habla m\'{a}s arriba.

\textbf{DESARROLLO:}

Interferencias en rendijas:

Simplemente hay que alinear (con precisi\'{o}n, eso s\'{i}), el laser con la
lamina con el numero de rendijas deseado y colocar la pantalla. Cuando los
elementos est\'{e}n alineados variamos la distancia de la pantalla a la
l\'{a}mina con las rendijas y medimos la separaci\'{o}n entre m\'{a}ximos.
Tomando la precauci\'{o}n de que la pantalla est\'{e} perpendicular al haz
de luz.

\vspace{1pt}Interfer\'{o}metro de Michelson:

El primer paso es alinear los elementos, en este caso se requiere m\'{a}s
precisi\'{o}n que en el caso anterior. Alineamos primero el laser con la
lente, colocamos entonces el espejo regulable de tal modo que se superpongan
los haces que se reflejan de cada espejo. Regulando el espejo con mucho
cuidado (y esperando cada vez que se mueve para eliminar las vibraciones)
acabaremos por obtener las manchas de luz correspondientes al patr\'{o}n de
interferencia de Michelson.

\vspace{1pt}

\textbf{RESULTADOS: }

-Separaci\'{o}n entre rendijas:

Primero mostramos los patrones que se han obtenido:

\qquad -1 rendija:

\FRAME{dtbpF}{469pt}{54.25pt}{0pt}{}{}{Figure }{\special{language
"Scientific Word";type "GRAPHIC";maintain-aspect-ratio TRUE;display
"USEDEF";valid_file "T";width 469pt;height 54.25pt;depth 0pt;original-width
464.5pt;original-height 51.9375pt;cropleft "0";croptop "1";cropright
"1";cropbottom "0";tempfilename 'IIWYBM04.wmf';tempfile-properties "XPR";}}

\qquad -2 rendijas:

\FRAME{dtbpF}{506.1875pt}{56.6875pt}{0pt}{}{}{Figure }{\special{language
"Scientific Word";type "GRAPHIC";maintain-aspect-ratio TRUE;display
"USEDEF";valid_file "T";width 506.1875pt;height 56.6875pt;depth
0pt;original-width 862pt;original-height 94.125pt;cropleft "0";croptop
"1";cropright "1";cropbottom "0";tempfilename
'IIWYCR05.wmf';tempfile-properties "XPR";}}

\qquad -3 rendijas:

\FRAME{dtbpF}{608.5pt}{47.5625pt}{0pt}{}{}{Figure }{\special{language
"Scientific Word";type "GRAPHIC";maintain-aspect-ratio TRUE;display
"USEDEF";valid_file "T";width 608.5pt;height 47.5625pt;depth
0pt;original-width 892.0625pt;original-height 67pt;cropleft "0";croptop
"1";cropright "1";cropbottom "0";tempfilename
'IIWYEE06.wmf';tempfile-properties "XPR";}}

\qquad -5 rendijas:

\FRAME{dtbpF}{608.5625pt}{44.125pt}{0pt}{}{}{Figure }{\special{language
"Scientific Word";type "GRAPHIC";maintain-aspect-ratio TRUE;display
"USEDEF";valid_file "T";width 608.5625pt;height 44.125pt;depth
0pt;original-width 834.875pt;original-height 57.9375pt;cropleft "0";croptop
"1";cropright "1";cropbottom "0";tempfilename
'IIWYFT07.wmf';tempfile-properties "XPR";}}

$
\begin{array}{lllllllllllllll}
\Delta x(mm) & 2.6 & 2.0 & 2.6 & 2.3 & 3.0 & 2.0 & 1.8 & 3.0 & 2.6 & 5.2 & 
6.2 & 8.2 & 7.8 & 3.0 \\ 
D(m) & 0.81 & 0.815 & 0.738 & 0.694 & 0.642 & 0.592 & 0.542 & 0.868 & 0.779
& 1.462 & 1.87 & 2.325 & 2.193 & 0.883
\end{array}
$

Para obtener $\frac{h}{\lambda }$ podemos realizar un ajuste a una linea
recta:

Obtenemos el valor: $\frac{h}{\lambda }=280\pm 10$, pero lo que se busca es
el valor de h. Al emplear una $\lambda $ conocida, se despeja h teniendo al
final:

$h=182\pm 7\mu m$

-Obtener la longitud de onda de un laser determinado:

\vspace{1pt}$
\begin{array}{lllllllllll}
\Delta x(mm) & 7.0 & 6.0 & 5.0 & 4.6 & 4.0 & 3.6 & 3.5 & 2.5 & 2.0 & 1.8 \\ 
D(m) & 1.92 & 1.68 & 1.53 & 1.275 & 1.260 & 1.050 & 0.98 & 0.64 & 0.76 & 0.50
\end{array}
$

El procedimiento es similar al del apartado anterior, solo que ahora lo que
ha de obtener es $\lambda $.

Del ajuste a una linea recta obtenemos: $\frac{h}{\lambda }=260\pm 20$, $%
h=700\pm 70nm$

\vspace{1pt}

-Interfer\'{o}metro de Michelson:

Patrones obtenidos:

\qquad -Maximo central:

\FRAME{dtbpF}{356.0625pt}{272.6875pt}{0pt}{}{}{Figure }{\special{language
"Scientific Word";type "GRAPHIC";maintain-aspect-ratio TRUE;display
"USEDEF";valid_file "T";width 356.0625pt;height 272.6875pt;depth
0pt;original-width 934.25pt;original-height 714.4375pt;cropleft "0";croptop
"1";cropright "1";cropbottom "0";tempfilename
'IIWYIE08.wmf';tempfile-properties "XPR";}}

\qquad -M\'{i}nimo central:

\FRAME{dtbpF}{371.5625pt}{255.75pt}{0pt}{}{}{Figure }{\special{language
"Scientific Word";type "GRAPHIC";maintain-aspect-ratio TRUE;display
"USEDEF";valid_file "T";width 371.5625pt;height 255.75pt;depth
0pt;original-width 750.5625pt;original-height 515.6875pt;cropleft
"0";croptop "1";cropright "1";cropbottom "0";tempfilename
'IIWYJS09.wmf';tempfile-properties "XPR";}}

\vspace{1pt}

CONCLUSIONES:

-El m\'{e}todo que hemos usado para medir d y $\lambda $ a partir del
patr\'{o}n de interferencias de las rendijas, ha demostrado ser m\'{a}s
fiable para medir d.

-Existen unos errores relativos un tanto elevados, se pueden anotar entre
las distintas causas el hecho de que cuando medimos distancias los
m\'{a}ximos no est\'{a}n marcados con precisi\'{o}n sino que van
difuninandose. Por lo que al medir la distancia entre varios m\'{a}ximos no
se puede tomar puntos de referencia con precisi\'{o}n, lo que introduce un
error mayor que el instrumental.

\end{document}
