%% This document created by Scientific Notebook (R) Version 3.0


\documentclass[12pt,thmsa]{article}
%%%%%%%%%%%%%%%%%%%%%%%%%%%%%%%%%%%%%%%%%%%%%%%%%%%%%%%%%%%%%%%%%%%%%%%%%%%%%%%%%%%%%%%%%%%%%%%%%%%%%%%%%%%%%%%%%%%%%%%%%%%%
\usepackage{sw20jart}

%TCIDATA{TCIstyle=article/art4.lat,jart,sw20jart}

%TCIDATA{<META NAME="GraphicsSave" CONTENT="32">}
%TCIDATA{Created=Mon Aug 19 14:52:24 1996}
%TCIDATA{LastRevised=Fri Jun 03 00:13:01 2005}
%TCIDATA{CSTFile=Lab Report.cst}
%TCIDATA{PageSetup=72,72,72,72,0}
%TCIDATA{AllPages=
%F=36,\PARA{038<p type="texpara" tag="Body Text" >\hfill \thepage}
%}


\input{tcilatex}
\begin{document}


\subsubsection{\protect\vspace{1pt}Sebasti\'{a}n Torrente Carrillo}

\subsubsection{8-6-stc}

\subsubsection{\protect\vspace{1pt}En el documento CO2HAWAI tenemos las
concentraciones de CO2 en la atm\'{o}sfera medida mes a mes, desde 1959 a
2002. Representar graficamente los datos. Aplicar la transformada de Fourier
para buscar una periodicidad a los datos. Aplicar la transformada de Fourier
a los datos sin la componente lineal.}

Para representar gr\'{a}ficamente primero tenemos que disponer los datos en
dos columnas de datos, meses para el eje x y la concentraci\'{o}n de CO2
para la y. Para ello se puede hacer o a mano (algo muy poco recomendable
debido a la cantidad de datos) o mediante un programa que lea y reordene.

Una vez reordenados los datos podemos representarlos con facilidad, para la
gr\'{a}fica he suprimido los datos correspondientes a los valores -99, ya
que no son significativos y hace que la gr\'{a}fica se vea a un tama\~{n}o
muy reducido si los representamos.

La gr\'{a}fica que obtenemos es la siguiente:

\FRAME{dtbpF}{476.3125pt}{320.1875pt}{0pt}{}{}{Figure }{\special{language
"Scientific Word";type "GRAPHIC";maintain-aspect-ratio TRUE;display
"USEDEF";valid_file "T";width 476.3125pt;height 320.1875pt;depth
0pt;original-width 572.125pt;original-height 383.9375pt;cropleft "0";croptop
"1";cropright "1";cropbottom "0";tempfilename
'IHH6KQ03.wmf';tempfile-properties "XPR";}}

Podemos observar que se compone de dos comportamientos, uno oscilatorio y el
otro lineal. Por lo que si quisiesemos realizar la transformada solo al
comportamiento oscilatorio tendriamos que eliminar el lineal. Para ello
podemos realizar un ajuste lineal y restarle la pendiente a todos los
puntos. Para la realizaci\'{o}n del ajuste he decidido emplear el programa
Microcal ORIGIN, por varios motivos, los principales son que maneja muy bien
los ajustes lineales y que me permite realizar operaciones sobre todos los
elementos de una columna.

\vspace{1pt}

Represento a continuaci\'{o}n la recta ajustada sobre los datos:

\vspace{1pt}\FRAME{dtbpF}{417.625pt}{280.75pt}{0pt}{}{}{Figure }{\special%
{language "Scientific Word";type "GRAPHIC";maintain-aspect-ratio
TRUE;display "USEDEF";valid_file "T";width 417.625pt;height 280.75pt;depth
0pt;original-width 572.125pt;original-height 383.9375pt;cropleft "0";croptop
"1";cropright "1";cropbottom "0";tempfilename
'IHH6KQ05.wmf';tempfile-properties "XPR";}}

Los datos que he obtenido para la recta son: $y=a\cdot x+b$ ; $a=0.11276$ ; $%
b=311.265$

\vspace{1pt}

Ahora aplicamos la transformada de Fourier para encontrar la periodicidad en
estos datos, tras aplicarla a los datos ''en bruto'' (esto es, sin quitar la
componente lineal) tenemos la siguiente gr\'{a}fica:

\FRAME{dtbpF}{420.625pt}{282.75pt}{0pt}{}{}{Figure }{\special{language
"Scientific Word";type "GRAPHIC";maintain-aspect-ratio TRUE;display
"USEDEF";valid_file "T";width 420.625pt;height 282.75pt;depth
0pt;original-width 572.125pt;original-height 383.9375pt;cropleft "0";croptop
"1";cropright "1";cropbottom "0";tempfilename
'IHH6KQ04.wmf';tempfile-properties "XPR";}}

Vemos que tenemos un pico muy alto en cerca del origen y otro en el punto
0.082, lo que dar\'{i}a una periodicidad de 12,2 meses.

\vspace{1pt}

Si aplicamos la transformada de Fourier a los datos ''refinados''
(quitandoles la componente lineal), tenemos:

\FRAME{dtbpF}{465.8125pt}{313.125pt}{0pt}{}{}{Figure }{\special{language
"Scientific Word";type "GRAPHIC";maintain-aspect-ratio TRUE;display
"USEDEF";valid_file "T";width 465.8125pt;height 313.125pt;depth
0pt;original-width 572.125pt;original-height 383.9375pt;cropleft "0";croptop
"1";cropright "1";cropbottom "0";tempfilename
'IHH6KQ06.wmf';tempfile-properties "XPR";}}

Vemos que tenemos un pico principal y dos picos secundarios. La posici\'{o}n
del pico principal es 0.0839844, lo que nos da una frecuencia de 11.9 meses
(cercana al a\~{n}o). Hay que destacar que la posici\'{o}n de este pico es
similar a la posici\'{o}n del pico secundario que se obtiene en el caso
anterior. Lo que me lleva a pensar que se trata del mismo pico, y que la
dependencia lineal restaba precisi\'{o}n a la transformada de Fourier. Si
esto ultimo es cierto podemos achacar las diferencias entre los valores de
las frecuencias al no poder ajustar la funci\'{o}n perfectamente a una linea
recta.

\vspace{1pt}

Incluyo en esta versi\'{o}n la gr\'{a}fica de la concentraci\'{o}n de CO2
sin la componente lineal:

\FRAME{dtbpF}{389pt}{261.5625pt}{0pt}{}{}{Figure }{\special{language
"Scientific Word";type "GRAPHIC";maintain-aspect-ratio TRUE;display
"USEDEF";valid_file "T";width 389pt;height 261.5625pt;depth
0pt;original-width 572.125pt;original-height 383.9375pt;cropleft "0";croptop
"1";cropright "1";cropbottom "0";tempfilename
'IHH9OA00.wmf';tempfile-properties "XPR";}}

\vspace{1pt}

Las dem\'{a}s anotaciones a este ejercicio son relativas al programa y son
comentadas en este.

\end{document}
