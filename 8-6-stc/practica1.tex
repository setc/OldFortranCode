%% This document created by Scientific Notebook (R) Version 3.0


\documentclass[12pt,thmsa]{article}
\usepackage{amssymb}

%%%%%%%%%%%%%%%%%%%%%%%%%%%%%%%%%%%%%%%%%%%%%%%%%%%%%%%%%%%%%%%%%%%%%%%%%%%%%%%%%%%%%%%%%%%%%%%%%%%%
\usepackage{sw20jart}

%TCIDATA{TCIstyle=article/art4.lat,jart,sw20jart}

%TCIDATA{<META NAME="GraphicsSave" CONTENT="32">}
%TCIDATA{Created=Mon Aug 19 14:52:24 1996}
%TCIDATA{LastRevised=Thu Jun 30 16:37:20 2005}
%TCIDATA{CSTFile=Lab Report.cst}
%TCIDATA{PageSetup=72,72,72,72,0}
%TCIDATA{AllPages=
%F=36,\PARA{038<p type="texpara" tag="Body Text" >\hfill \thepage}
%}


\input{tcilatex}
\begin{document}


\subsubsection{\protect\vspace{1pt}Pr\'{a}ctica I : Polarizaci\'{o}n.}

\subsubsection{Sebasti\'{a}n Torrente Carrillo}

\vspace{1pt}

\textbf{OBJETIVOS:}

i ) Medir y representar gr\'{a}ficamente la intensidad de luz reflejada en
un vidrio en funci\'{o}n del \'{a}ngulo de incidencia para la TM

ii ) Determinar el \'{a}ngulo de Brewster y estimar el \'{i}ndice de
refracci\'{o}n del vidrio utilizado.

iii ) Obtener el \'{i}ndice de refracci\'{o}n del vidrio por un m\'{e}todo
alternativo y comparar los resultados obtenidos para el apartado anterior.

iv ) Representar la intensidad transmitida por los polarizadores lineales en
funci\'{o}n del \'{a}ngulo relativo de sus ejes completando un giro completo
de uno de los polarizadores. Se\~{n}alar los m\'{i}nimos que se obtienen.

v ) Generar luz polarizada lineal y medir su vector de Stokes.

vi ) Generar luz polarizada circular y medir su vector de Stokes.

vii ) Estimar el vector de Stokes de la luz que emite uno de los laseres.

vii ) Construir una trampa de luz mediante elementos apropiados de
polarizaci\'{o}n.

\vspace{1pt}

\textbf{BASE TE\'{O}RICA:}

\vspace{1pt}

Usando las ecuaciones de Fresnell tenemos que la onda reflejada desaparece
si $\theta _{1}+\theta _{2}=\frac{\pi }{2}$ , la expresi\'{o}n del angulo de
Brewster ser\'{a} entonces $\theta _{B}=\arctan (\frac{n}{n^{\prime }})$. Si
medimos hasta encontrar un angulo en el que no se refleja luz obtendremos el
\'{a}ngulo de Brewster. Sustituyendo en la expresi\'{o}n correspondiente a
este podremos despejar $n^{\prime }$.

\vspace{1pt}

Otro m\'{e}todo para poder medir $n^{\prime }$ es medir el factor de
reflexi\'{o}n de la l\'{a}mina en incidencia normal. Teniendo en cuenta que
la luz pasa por dos interfases, se puede despejar $n^{\prime }$ de la
ecuaci\'{o}n del factor de transmisi\'{o}n (poner)

\vspace{1pt}

Y por \'{u}ltimo podemos comprobar la ley de malus empleando un polarizador
y un analizador. Recordemos que seg\'{u}n la ley de Malus la intensidad de
la luz sigue la proporci\'{o}n: $I\varpropto \cos ^{2}(\theta )$

\vspace{1pt}

\textbf{MONTAJE\ EXPERIMENTAL: }

Consta de los siguientes elementos:

$\cdot$ 1 l\'{a}ser de $\lambda $ = 670 nm

$\cdot$ 1 banco \'{o}ptico

$\cdot$ 2 polarizadores lineales

$\cdot $ 2 l\'{a}minas de $\frac{\pi }{4}$

$\cdot$ 1 espejo plano

$\cdot $ 1 l\'{a}mina plano paralela

\vspace{1pt}

\textbf{DESARROLLO:}

\vspace{1pt}Obtenci\'{o}n del \'{a}ngulo de Brewster:

Para medir la intensidad de la luz reflejada en funci\'{o}n del \'{a}ngulo
de incidencia, dispondremos los siguientes elemento sen el banco
\'{o}pticao: un polarizador con su linea neutra en posici\'{o}n horizontal.
Detr\'{a}s colocaremos una l\'{a}mina plano-paralela. As\'{i} se obtiene luz
polarizada horzontal. Con el soporte adecuado sobre la l\'{a}mina podremos
medir el \'{a}ngulo del l\'{a}mina respecto al polarizador gracias a la
rueda graduada.

Para medir la intensidad reflejada para cada \'{a}ngulo emplearemos el
lux\'{o}metro. Se puede comprobar que para distintas incidencias, la
intensidad se mantiene estable. Para el apartado correspondiente a la
obtenci\'{o}n del \'{a}ngulo de Brewster tomaremos medidas en intervalos de
10$\UNICODE[m]{0xb0}$ hasta acercarnos al \'{a}ngulo de Brewster, donte se
toman intervalos de medidas m\'{a}s peque\~{n}os, para poder acotar mejor
dicho \'{a}ngulo.

\vspace{1pt}

Hallar el \'{i}ndice de refracci\'{o}n del vidrio:

Tenemos, a partir de las leyes de Fresner la expresi\'{o}n:

$T=\frac{4nn^{\prime }}{(n+n^{\prime })^{2}}$

Con n indice del primer medio y n' \'{i}ndice de refracci\'{o}n del segundo
medio.

Para la primera interfase (n=1): $T_{1}=\frac{4n^{\prime }}{(1+n^{\prime
})^{2}}$ y para la segunda interfase: $T_{2}=\frac{4n^{\prime }}{(n^{\prime
}+1)^{2}}$

Empleando las intensidades $I_{0}$ inicial, $I$ de la interfase, y $I_{f}$
final, tenemos las relaciones:

$I_{0}=T_{1}\cdot I$

$I=T_{2}\cdot I_{f}$

Desarrollamos: $I_{0}=T_{1}T_{2}I_{f}$ \ ; \ $\frac{I_{0}}{I_{f}}=\left( 
\frac{4n^{\prime }}{(1+n^{\prime })^{2}}\right) ^{2}$

En el laboratorio podemos medir las intensidades mediante el lux\'{o}metro,
para medir las intensidades iniciales y final hay que medir \ antes y
despues de la refracci\'{o}n. Y mediante el uso de la f\'{o}rmula que hemos
obtenido tendremos el indice de refracci\'{o}n de la l\'{a}mina n'.

\vspace{1pt}

Comprobaci\'{o}n de la ley de Malus y obtenci\'{o}n de par\'{a}metros de
Stokes:

Se puede comprobar esta ley empleando dos polaridadores lineales, \ iremos
orientandolos en distintiso \'{a}ngulos relativos entre sus lineas neutras y
medimos la intensidad de la luz que pasa ambos polarizadores.

Para generar los distintos tipos de luz polarizada se realizan los
siguientes montajes:

Luz polarizada horizontal: Situar la linea neutra del polarizador en
horizontal, y con un analizador realizaremos las medidas necesarias para
medir los parametros de Stokes correspondientes a la luz horizontal y
vertical, y el componente correspondiente a la orientaci\'{o}n a +45$%
\UNICODE[m]{0xb0}$ o -45$\UNICODE[m]{0xb0}$ . Necesitaremos tambi\'{e}n un
polarizador circular (que se puede construir facilemnte con un polarizador
lineal inclinado 45 grados y una lamina de cuarto de onda). Sin embargo para
la luz que empleamos (un laser) el retardadro de cuarto de onda no se
comportara como tal, sino que introducir\'{a} un desfase de aproximadamente
0.61p, por lo que la luz no ser\'{a} polarizada circular, dando lugar a un
error en los parametros de Stokes. A esto se a\~{n}ada el hecho de que el
polarizador no es perfecto.

Para el caso de luz polarizada circular dextrogira, emplearemos el montaje
que se ha rese\~{n}ado antes, un polarizador lineal inclinado en un
\'{a}ngulo de +45 grados seguida de una l\'{a}mina de cuarto de honda,
obtendremos una luz ligeramente el\'{i}ptica, por lo que los elementos del
vector de Stokes no ser\'{a}n los mismo que los de una luz polarizada
circular dextrogira ideal.

\vspace{1pt}

\textbf{RESULTADOS:}

Comprobaci\'{o}n de la Ley de Brewster:

$
\begin{array}{lllllllllllll}
\theta (\UNICODE[m]{0xb0}) & 10 & 20 & 30 & 40 & 50 & 55 & 56 & 57 & 58 & 60
& 70 & 80 \\ 
I(ua) & 33 & 26 & 22 & 14 & 4 & 1 & 0 & 1 & 1 & 2 & 10 & 23
\end{array}
$

$\theta _{B}=56\pm 2\UNICODE[m]{0xb0}$

Para obtener el indice de refracci\'{o}n del vidrio empleamos la ley de
Snell:

$n_{vidrio}\cdot 1=n_{aire}\cdot \sin (\theta _{B})$ \ $\Longrightarrow $ $%
n_{vidrio}=1.5\pm 0.1$

Intensidad de luz transmitida en incidencia normal:

Intensidad incidente: $I_{incidente}=89\pm 5ua$

Intensidad transmitida: $I_{trans}=51\pm 5ua$

\vspace{1pt}

Aplicando: $\frac{I_{0}}{I_{f}}=\left( \frac{4n^{\prime }}{(1+n^{\prime
})^{2}}\right) ^{2}$, despejando n' queda la ecuaci\'{o}n: $n^{\prime }=%
\frac{-(2-4\sqrt{\frac{I_{trans}}{I_{trans}}})\pm \sqrt{(2-4\sqrt{\frac{%
I_{trans}}{I_{trans}}})^{2}-4}}{2}\simeq 1.7\pm 0.3$

\vspace{1pt}

Comprobaci\'{o}n de la ley de Malus:

$
\begin{array}{llllllllllllllllllllllllll}
\theta (\UNICODE[m]{0xb0}) & 0 & 15 & 30 & 45 & 60 & 75 & 90 & 105 & 120 & 
135 & 150 & 165 & 180 & -15 & -30 & -45 & -60 & -75 & -90 & -105 & -120 & 
-135 & -150 & -165 & -180 \\ 
I(ua) & 36 & 35 & 29 & 21 & 12 & 4 & 0 & 1 & 7 & 15 & 25 & 32 & 36 & 33 & 26
& 17 & 8 & 2 & 0 & 3 & 9 & 19 & 29 & 35 & 35
\end{array}
$

\vspace{1pt}

Obtenci\'{o}n de los par\'{a}metros de Stokes:

\vspace{1pt}

Luz polarizada horizontal:

Luz inicial: $I=59ua$

Con polarizador colocado en horizontal: $I_{h}=39ua$

Con polarizador colocado en vertical: $I_{v}=0ua$

Con polarizador inclinado +45$\UNICODE[m]{0xb0}$: $I_{+45}=21ua$

Con polarizador inclinado -45$\UNICODE[m]{0xb0}$: $I_{-45}=20ua$

Luz polarizada circular dextrogira: $I_{cd}=18ua$

Luz polarizada circular lev\'{o}gira: $I_{cl}=19ua$

\vspace{1pt}

Para estos datos tenemos un vector de Stokes: $\left( 
\begin{array}{l}
58 \\ 
-39 \\ 
0 \\ 
-1
\end{array}
\right) $: podemos dividir entre 58 para normalizarlo $\left( 
\begin{array}{l}
1 \\ 
-0.67 \\ 
0 \\ 
-0.02
\end{array}
\right) $

Luz polarizada circular (dextr\'{o}gira):

$I=28ua$ ; $I_{h}=11ua$ ; $I_{v}=8ua$ ; $I_{45}=13ua$ ; $I_{-45}=17ua$ ; $%
I_{cd}=12ua$ ; $I_{cl}=6ua$

Vector de Stokes:  $\left( 
\begin{array}{l}
28 \\ 
3 \\ 
6 \\ 
6
\end{array}
\right) $: normalizado $\left( 
\begin{array}{l}
1 \\ 
0.1 \\ 
0.2 \\ 
0.2
\end{array}
\right) $

Luz procedente de un laser:

$I=90ua$ ; $I_{h}=2ua$ ; $I_{v}=58ua$ ; $I_{45}=32ua$ ; $I_{-45}=30ua$ ; $%
I_{cd}=25ua$ ; $I_{cl}=21ua$

Vector de Stokes: $\left( 
\begin{array}{l}
90 \\ 
-54 \\ 
2 \\ 
4
\end{array}
\right) $: normalizado $\left( 
\begin{array}{l}
1 \\ 
-0.62 \\ 
0.02 \\ 
0.04
\end{array}
\right) $

CONCLUSIONES:

\vspace{1pt}

-De los dos m\'{e}todos sugeridos en la pr\'{a}ctica para la obtenci\'{o}n
del indice de refracci\'{o}n en una l\'{a}mina, es preferible emplear el m%
\'{e}todo en el que se mide el \'{a}ngulo de Brewster y a partir de este se
obtiene n, debido a su mayor facilidad de ejecuci\'{o}n, y se puede obtener
el \'{a}ngulo de Brewster con bastante precisi\'{o}n.

-La comprobaci\'{o}n de la ley de Malus es (como se puede apreciar en la gr%
\'{a}fica correspondiente) satisfactoria.

-De la obtenci\'{o}n de parametros de Stokes de la luz polarizada, se puede
ver que las dos luces de polarizaci\'{o}n conocida dan resultados
razonablemente buenos. En cuanto al laser, se puede ver en sus par\'{a}%
metros que se trata de una mezcla de luces polarizada, pero que predom\'{i}%
na el t\'{e}rmino vertical.

-Un apunte acerca de los lux\'{o}metros, son muy sensibles al \'{a}ngulo con
el que la luz incide sobre ellos, eso es importante en el caso del laser, ya
que debido a la naturaleza del haz podemos experimentar una fuerte variaci%
\'{o}n en las medidas si el haz no incide perpendicularmente.

\vspace{1pt}-Tambi\'{e}n hay que se\~{n}alar que los polarizadores y la l%
\'{a}mina no son perfectos, por lo que, al suponerlos ideales en los
resultados se tiene un margen de error debido a esto en los resultados.

\vspace{1pt}


\ 

\end{document}
