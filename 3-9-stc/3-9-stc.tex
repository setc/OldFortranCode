%% This document created by Scientific Notebook (R) Version 3.0


\documentclass[12pt,thmsa]{article}
%%%%%%%%%%%%%%%%%%%%%%%%%%%%%%%%%%%%%%%%%%%%%%%%%%%%%%%%%%%%%%%%%%%%%%%%%%%%%%%%%%%%%%%%%%%%%%%%%%%%%%%%%%%%%%%%%%%%%%%%%%%%
\usepackage{sw20jart}

%TCIDATA{TCIstyle=article/art4.lat,jart,sw20jart}

%TCIDATA{<META NAME="GraphicsSave" CONTENT="32">}
%TCIDATA{Created=Mon Aug 19 14:52:24 1996}
%TCIDATA{LastRevised=Mon Apr 18 00:38:19 2005}
%TCIDATA{CSTFile=Lab Report.cst}
%TCIDATA{PageSetup=72,72,72,72,0}
%TCIDATA{AllPages=
%F=36,\PARA{038<p type="texpara" tag="Body Text" >\hfill \thepage}
%}


\input{tcilatex}
\begin{document}


\subsection{\protect\vspace{1pt}Sebasti\'{a}n Torrente Carrillo}

\section{\textbf{Ejercicio 3-9:}}

\textbf{LA distribuci\'{o}n de velocidades moleculares en un gas idea en
equilibrio t\'{e}rmico a la temperatura T viene dad por la funci\'{o}n de
distribuci\'{o}n de Maxwell-Boltzmann f(v)}

$\mathbf{f(v)=4\pi (}\frac{m}{2\pi kT}\mathbf{)}^{\frac{3}{2}}\mathbf{v}%
^{2}\exp \mathbf{(-}\frac{m\frac{v^{2}}{2}}{kT}\mathbf{)}$

\textbf{con k cte de Boltzman, y m la masa de las molecula del gas.}

\textbf{Representa gr\'{a}ficamente esta funci\'{o}n para el gas arg\'{o}n y
las siguientes temperaturas:}

$\mathbf{T1=100K}$

$\mathbf{T2=200K}$

$\mathbf{T3=300K}$

\textbf{Y calcula la velocidad cuadr\'{a}tica media}

\textbf{\vspace{1pt}}

La representaci\'{o}n gr\'{a}fica no plantea muchos problemas, simplemente
planteamos la funci\'{o}n y mediante un bucle vamos obteniendo puntos que
luego graficaremos mediante el GNUplot. Las gr\'{a}ficas se incluyen en los
documentos de imagen de nombre ''3-9a-stc.gif'', ''3-9b-stc.gif'' y
''3-9-stc.gif'' para temperaturas T1, T2 y T3 respectivamente.

\vspace{1pt}

El apartado de inter\'{e}s de este ejercicio es el segundo. Sabemos, que
dada una distribuci\'{o}n continua de probabilidades f(x), la media de una
magnitud g(x) $<g(x)>$ se calcula mediante la expresi\'{o}n $%
<g(x)>=\int\limits_{-\infty }^{+\infty }g(x)\cdot f(x)^{2}dx$

En nuestro caso tenemos que obtener la velocidad cuadr\'{a}tica media, esto
es $<v^{2}>=\int\limits_{-\infty }^{+\infty }v^{2}\cdot f(v)^{2}dv$

Tenemos un problema, y es que los intervalos de esta integral son infinitos,
pero sabemos que la funci\'{o}n f(v) es convergente, luego su cuadrado
tambi\'{e}n lo \'{e}s. Solo quedar\'{i}a comprobar que $v^{2}\cdot f(v)^{2}$
converge. Si es as\'{i} para poder integrar la funci\'{o}n por m\'{e}todos
numericos cogeremos un intervalo de integraci\'{o}n lo suficientemente
grande de tal modo que aumentarlo variar\'{i}a el resultado en una cantidad
despreciable. En este caso concreto tenemos que la funci\'{o}n si es
convergente, por lo que la integral tiene sentido. Como se puede ver en la
gr\'{a}fica, esta consta de dos picos y luego decae al cero de modo brusco.

\vspace{1pt}

Todas las dem\'{a}s aclaraciones se encuentran en el c\'{o}digo fuente del
programa.

\end{document}
