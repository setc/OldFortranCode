%% This document created by Scientific Notebook (R) Version 3.0


\documentclass[12pt,thmsa]{article}
%%%%%%%%%%%%%%%%%%%%%%%%%%%%%%%%%%%%%%%%%%%%%%%%%%%%%%%%%%%%%%%%%%%%%%%%%%%%%%%%%%%%%%%%%%%%%%%%%%%%%%%%%%%%%%%%%%%%%%%%%%%%
\usepackage{sw20jart}

%TCIDATA{TCIstyle=article/art4.lat,jart,sw20jart}

%TCIDATA{<META NAME="GraphicsSave" CONTENT="32">}
%TCIDATA{Created=Mon Aug 19 14:52:24 1996}
%TCIDATA{LastRevised=Mon Apr 18 00:58:39 2005}
%TCIDATA{CSTFile=Lab Report.cst}
%TCIDATA{PageSetup=72,72,72,72,0}
%TCIDATA{AllPages=
%F=36,\PARA{038<p type="texpara" tag="Body Text" >\hfill \thepage}
%}


\input{tcilatex}
\begin{document}


\subsection{Ejercicio 3-8:}

\subsection{Calcula numericamente hasta la cuarta cifra significativa la
siguiente integral doble:}

\subsection{$I=\int\nolimits_{0}^{1}dx\int\nolimits_{x^{2}}^{x}dy\exp (xy)$}

Como vemos se trata de una integral doble, ya hemos tratado en ejercicios
anteriores este problema. Tendremos que integral la exponencial dos veces,
la primera integraci\'{o}n se har\'{a} con la variable $y$ y con l\'{i}mites
dependientes de x, y la segunda con respecto a x. La integral no ofrece
muchos problemas y podremos realizar el programa empleando partes del codigo
empleado en programas anteriores, haciendo pocos cambios.

\vspace{1pt}

Todas las dem\'{a}s consideraciones son comentadas en el codigo fuente del
programa.

\end{document}
