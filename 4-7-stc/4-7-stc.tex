%% This document created by Scientific Notebook (R) Version 3.0


\documentclass[12pt,thmsa]{article}
%%%%%%%%%%%%%%%%%%%%%%%%%%%%%%%%%%%%%%%%%%%%%%%%%%%%%%%%%%%%%%%%%%%%%%%%%%%%%%%%%%%%%%%%%%%%%%%%%%%%%%%%%%%%%%%%%%%%%%%%%%%%
\usepackage{sw20jart}

%TCIDATA{TCIstyle=article/art4.lat,jart,sw20jart}

%TCIDATA{<META NAME="GraphicsSave" CONTENT="32">}
%TCIDATA{Created=Mon Aug 19 14:52:24 1996}
%TCIDATA{LastRevised=Thu Apr 28 23:00:17 2005}
%TCIDATA{CSTFile=Lab Report.cst}
%TCIDATA{PageSetup=72,72,72,72,0}
%TCIDATA{AllPages=
%F=36,\PARA{038<p type="texpara" tag="Body Text" >\hfill \thepage}
%}


\input{tcilatex}
\begin{document}


\subsubsection{EJERCICIO 4.7}

\subsubsection{El la aproximaci\'{o}n de peque\~{n}as oscilaciones y
suponiendo una masa del resorte despreciable , la Lagrangiana para un
p\'{e}ndulo de WilderForce es:}

\subsubsection{$L=\frac{1}{2}m\left( \frac{dz}{dt}\right) ^{2}+\frac{1}{2}%
I\left( \frac{d\theta }{dt}\right) ^{2}-\frac{1}{2}kz^{2}-\frac{1}{2}\delta
\theta -\frac{1}{2}\epsilon z\theta $}

\subsubsection{Con m la masa que cuelga del pendulo, k y $\delta $ las
constantes el\'{a}sticas longitudinal y torsional, $\epsilon $ una constante
que describe el acoplamiento entre ambas oscilaciones .}

\subsubsection{\protect\vspace{1pt}}

a) Escribe las ecuaciones del movimiento de este sistema, resuelvelas y
representa z(t) y $\theta (t)$

\vspace{1pt}Para calcular las ecuaciones de movimiento usaremos el
formalismo de Lagrange. Las ecuaciones de movimiento se obtienen a partir de
las ecuaci\'{o}nes de lagrange, que en nuestro caso son dos:

$\frac{d}{dt}(\frac{\partial L}{\partial z^{\prime }})-\frac{\partial L}{%
\partial z}=0$

$\frac{d}{dt}(\frac{\partial L}{\partial \theta ^{\prime }})-\frac{\partial L%
}{\partial \theta }=0$

Con L la lagrangiana del sistema. Realizamos las derivadas de la Lagrangiana
necesarias.

$\frac{\partial L}{\partial z^{\prime }}=mz^{\prime }$ ; $\frac{\partial L}{%
\partial \theta ^{\prime }}=I\theta ^{\prime }$

$\frac{d}{dt}(\frac{\partial L}{\partial z^{\prime }})=mz^{\prime \prime }$
; $\frac{d}{dt}(\frac{\partial L}{\partial \theta ^{\prime }})=m\theta
^{\prime \prime }$

$\frac{\partial L}{\partial z}=-kz-\frac{1}{2}\epsilon \theta =-\frac{1}{2}%
(k+\epsilon \theta )$ ; $\frac{\partial L}{\partial \theta }=\delta \theta -%
\frac{1}{2}\epsilon z$

Obtenemos las ecuaciones:

$mz^{\prime \prime }+\frac{1}{2}(2k+\epsilon \theta )=0$ ; $z^{\prime \prime
}=-\frac{1}{2m}(2kz+\epsilon \theta )$

$I\theta ^{\prime \prime }+\delta \theta -\frac{1}{2}\epsilon z=0$ ; $\theta
^{\prime \prime }=-\frac{1}{2I}(2\delta z+\epsilon \theta )$

En el enunciado se nos dan las condiciones iniciales: $z(0)=0.01m$ y $\theta
(0)=0$ . Tenemos dos ecuaciones de segundo orden, por lo que necesitaremos
reducir el orden de las dos convirtiendolas en un sistema de ecuaciones
lineales.

$z^{\prime }(t)=u(t)$

$\theta ^{\prime }(t)=v(t)$

$u^{\prime }(t)=-\frac{1}{2m}(k+\epsilon \theta )$

$v^{\prime }(t)=\frac{1}{I}(\delta \theta -\frac{\epsilon z}{2})$

Que si podemos resolver numericamente, para ello emplearemos el m\'{e}todo
de Runge-Kutta, las funciones que resultan se muestran en las siguientes
imagenes:

\vspace{1pt}

z(t):

\FRAME{dtbpF}{631.75pt}{309.25pt}{0pt}{}{}{Figure }{\special{language
"Scientific Word";type "GRAPHIC";maintain-aspect-ratio TRUE;display
"USEDEF";valid_file "T";width 631.75pt;height 309.25pt;depth
0pt;original-width 626.3125pt;original-height 305.625pt;cropleft "0";croptop
"1";cropright "1";cropbottom "0";tempfilename
'IFOBBN00.wmf';tempfile-properties "XPR";}}

$\theta (t):$

\FRAME{dtbpF}{631.75pt}{309.25pt}{0pt}{}{}{Figure }{\special{language
"Scientific Word";type "GRAPHIC";maintain-aspect-ratio TRUE;display
"USEDEF";valid_file "T";width 631.75pt;height 309.25pt;depth
0pt;original-width 626.3125pt;original-height 305.625pt;cropleft "0";croptop
"1";cropright "1";cropbottom "0";tempfilename
'IFOBCJ01.wmf';tempfile-properties "XPR";}}

Representamos las dos funciones a la vez para comprobar que se alternan los
dos tipos de oscilaciones:

\FRAME{dtbpF}{631.75pt}{309.25pt}{0pt}{}{}{Figure }{\special{language
"Scientific Word";type "GRAPHIC";maintain-aspect-ratio TRUE;display
"USEDEF";valid_file "T";width 631.75pt;height 309.25pt;depth
0pt;original-width 626.3125pt;original-height 305.625pt;cropleft "0";croptop
"1";cropright "1";cropbottom "0";tempfilename
'IFOBDS02.wmf';tempfile-properties "XPR";}}

\vspace{1pt}

A partir de las ecuaciones de movimiento se puede comprobar el valor de las
frecuencias (se puede sacar ya que siguen la ecuaci\'{o}n de ondas), y se
puede comprobar como este coincide con el valor previsto por la teor\'{i}a.

\vspace{1pt}

Solo nos queda dar otros valores a las constantes $m,I,k,\delta $ y ver que
forma toman las curvas:

Las constantes elegidas son:

$m=1.5kg$ ; $I=4\cdot 10^{-4}$ ; $k=3.0$ ; $\delta =0.01$

Las gr\'{a}ficas que obtenemos una vez calculado por el programa son:

\FRAME{dtbpF}{631.75pt}{309.25pt}{0pt}{}{}{Figure }{\special{language
"Scientific Word";type "GRAPHIC";maintain-aspect-ratio TRUE;display
"USEDEF";valid_file "T";width 631.75pt;height 309.25pt;depth
0pt;original-width 626.3125pt;original-height 305.625pt;cropleft "0";croptop
"1";cropright "1";cropbottom "0";tempfilename
'IFOCV905.wmf';tempfile-properties "XPR";}}

Lor primero que podemos apreciar es que la oscilaci\'{o}n longitudinal se
mantiene constante para toda la oscilaci\'{o}n. En el caso de la oscilaci%
\'{o}n c\'{i}clica vemos que si bien no tiene un esquema sinuidal puro como
el de la oscilaci\'{o}n longitudinal si que se mantiene constante. Lo m\'{a}%
s importante es ver que en este caso las dos oscilaciones no se alternan
como en el caso anterior, manteniendo una cierta ''independencia'' entre
ellas.

\vspace{1pt}

Las dem\'{a}s consideraciones sobre este problema son referentes a la
programaci\'{o}n de este, por lo que se pueden encontrar en el codigo fuente
del ejercicio.

\vspace{1pt}

\end{document}
