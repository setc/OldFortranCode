%% This document created by Scientific Notebook (R) Version 3.0


\documentclass[12pt,thmsa]{article}
%%%%%%%%%%%%%%%%%%%%%%%%%%%%%%%%%%%%%%%%%%%%%%%%%%%%%%%%%%%%%%%%%%%%%%%%%%%%%%%%%%%%%%%%%%%%%%%%%%%%%%%%%%%%%%%%%%%%%%%%%%%%
\usepackage{sw20jart}

%TCIDATA{TCIstyle=article/art4.lat,jart,sw20jart}

%TCIDATA{<META NAME="GraphicsSave" CONTENT="32">}
%TCIDATA{Created=Mon Aug 19 14:52:24 1996}
%TCIDATA{LastRevised=Mon Apr 04 00:13:38 2005}
%TCIDATA{CSTFile=Lab Report.cst}
%TCIDATA{PageSetup=72,72,72,72,0}
%TCIDATA{AllPages=
%F=36,\PARA{038<p type="texpara" tag="Body Text" >\hfill \thepage}
%}


\input{tcilatex}
\begin{document}


\subsection{2--6-stc}

\subsection{Sebasti\'{a}n Torrente Carrillo:}

\vspace{1pt}\textbf{Calcular la resistencia equivalente en un sistema de
redes dado:}

El sistema que se nos propone en el enunciado es el siguiente:

\FRAME{dtbpF}{185.9375pt}{181.4375pt}{0pt}{}{}{Figure }{\special{language
"Scientific Word";type "GRAPHIC";maintain-aspect-ratio TRUE;display
"USEDEF";valid_file "T";width 185.9375pt;height 181.4375pt;depth
0pt;original-width 182.9375pt;original-height 178.4375pt;cropleft
"0";croptop "1";cropright "1";cropbottom "0";tempfilename
'IEE5QQ01.wmf';tempfile-properties "XPR";}}

Y se nos pide calcular la resistencia equivalente entre los puntos a y b. Se
nos da como dato que la resistencia por unidad de longitud es de 5 ohmnios
por centimetro y d=10cm.

Para calcular la resistencia equivalente primero tendremos que calcular
cuanto vale la resistencia en cada tramo. Se puede ver con facilidad que
existen tres tipos de tramos distintos. Los tramos de longitud d/2 que se
conectan con los puntos a \'{o} b. Los tramos de longitud d que se
encuentran entre 1-4 y 2-3. Y los tramos correspondientes con los lados del
cuadrado inscrito.

Calculamos los valores de la resistencia para cada tipo de tramo, al
tratarse de calculos muy simples (multiplicar la longitud del tramo por la
resistencia por centimetro) los obviamos y damos directamente los resultados:

Resistencia tipo 1 (res1 en el c\'{o}digo fuente): 50 ohmnios

Resistencia tipo 2 (res2 en el c\'{o}digo fuente): 100 ohmnios

Resistencia tipo 3 (res3 en el c\'{o}digo fuente): 70,71068

\vspace{1pt}

Para calcular la resistencia tenemos que tener en cuenta que la diferencia
de voltaje entre dos puntos con una resistencia entre estos viene dada por: $%
V=I\cdot R$

En nuestro caso queremos calcular la resistencia, seg\'{u}n el enunciado
tiene un valor constante, por lo que para un valor dado de $V$, y conociendo
cuando vale cada tramo de la resistencia podemos calcular $I$ y con los dos
valores podemos calcular de forma muy simple el valor de la resistencia
equivalente. Para todo este proceso necesitaremos un valor arbitrario de V,
una vez elegido planteamos un sistema de ecuaciones lineales para calcular
el vector con los valores de las intensidades. Una vez resuelto tendremos
por un lado el valor de V y por otro el de I, por lo que ya podremos obtener
R.

\vspace{1pt}

El resto de anotaciones ya corresponden al programa en s\'{i}, por lo que se
encuentran en el c\'{o}digo fuente de este como comentarios.

\vspace{1pt}

A continuaci\'{o}n expongo la matriz correspondiente al problema:

$\left( 
\begin{array}{lllllllllll}
1 & 1 & 0 & 0 & 0 & 0 & 0 & 0 & 0 & 0 & -1 \\ 
-1 & 0 & 1 & 1 & 0 & 1 & 0 & 0 & 0 & 0 & 0 \\ 
0 & -1 & 0 & -1 & -1 & 0 & -1 & 0 & 0 & 0 & 0 \\ 
0 & 0 & 0 & 0 & -1 & 0 & 1 & -1 & 0 & -1 & 0 \\ 
0 & 0 & -1 & 0 & 0 & -1 & 0 & 1 & 1 & 0 & 0 \\ 
R_{1} & -R_{2} & 0 & R_{4} & 0 & 0 & 0 & 0 & 0 & 0 & 0 \\ 
0 & 0 & R_{3} & 0 & 0 & -R_{6} & 0 & 0 & 0 & 0 & 0 \\ 
0 & 0 & 0 & -R_{4} & 0 & R_{6} & R_{7} & R_{8} & 0 & 0 & 0 \\ 
0 & 0 & 0 & 0 & 0 & 0 & 0 & R_{8} & R_{9} & R_{10} & 0 \\ 
0 & R_{2} & 0 & 0 & R_{5} & 0 & 0 & 0 & 0 & R_{10} & 0 \\ 
0 & 0 & 0 & 0 & R_{5} & 0 & R_{7} & 0 & 0 & 0 & 0
\end{array}
\right) \cdot \left( 
\begin{array}{l}
I_{1} \\ 
I_{2} \\ 
I_{3} \\ 
I_{4} \\ 
I_{5} \\ 
I_{6} \\ 
I_{7} \\ 
I_{8} \\ 
I_{9} \\ 
I_{10} \\ 
I_{11}
\end{array}
\right) =\left( 
\begin{array}{l}
0 \\ 
0 \\ 
0 \\ 
0 \\ 
0 \\ 
0 \\ 
0 \\ 
0 \\ 
0 \\ 
0 \\ 
V
\end{array}
\right) $

\end{document}
